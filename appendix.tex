\section{Technical appendix: omitted proofs}
\label{sect:proofs}

The enumeration of lemmas already stated in the body of the article is unchanged.

\setcounter{lemmaAppendix}{\value{lem:confluence-perm}}
\begin{lemmaAppendix}[Confluence of $\rw_\perm$]
	\label{lemmaAppendix:confluence-perm}
	Reduction $\rw_\perm$ is confluent.
\end{lemmaAppendix}

\begin{proof}
	We prove local confluence; by Newman's lemma and strong normalization of $\rw_\perm$ (\Cref{lemma:strong-normalization}), confluence follows. We consider each reduction rule against those lower down in Figure~\ref{fig:reduction rules}. For the symmetric rule pairs $\cancelL$/$\cancelR$, $\plusL$/$\plusR$, and $\plusFun/\plusArg$ we present only the first case. Unless otherwise specified, we let $a<b<c$.
	
	\newcommand\itm[2]{\medskip\noindent(#1)}% \qquad$#2$}
	
	%=====
	\itm\idem{\trm{N +a N}\rw N}
	{\small
		\[
		\vc{\begin{tikzpicture}[x=20pt,y=1.5ex]
			\node[anchor=base east] (a) at (0,0) {$\trm{(N+aN)+aM}$};
			\node[anchor=base west] (b) at (1,0) {$\trm{N+aM}$};
			\draw[rw] (0,1) --node[above]{$\idem$}    (1,1);
			\draw[rw] (0,0) --node[below]{$\cancelL$} (1,0);
			\end{tikzpicture}}
		\]
		\[
		\vcenter{\hbox{\begin{tikzpicture}
				\matrix [matrix of math nodes] (m) {
					\trm{(N+aM)+a(N+aM)} &[20pt] \trm{N+aM}
					\\[20pt]\trm{N+a(N+aM)}
					\\ };
				\draw[rw] (m-1-1) --node[above]{$\idem$}    (m-1-2);
				\draw[rw] (m-1-1) --node[left] {$\cancelL$} (m-2-1);
				\draw[rw,implied] (m-2-1) --node[below right=-2pt] {$\cancelR$} (m-1-2);
				\end{tikzpicture}}}
		\qquad
		\vcenter{\hbox{\begin{tikzpicture}
				\matrix [matrix of math nodes] (m) {
					\trm{C[N+aN]}    &[20pt] C[N]
					\\[20pt]\trm{C[N]+aC[N]}
					\\ };
				\draw[rw] (m-1-1) --node[above]{$\idem$}  (m-1-2);
				\draw[rw] (m-1-1) --node[left] {$\plusX$} (m-2-1);
				\draw[rw,implied] (m-2-1) --node[below right=-2pt] {$\idem$} (m-1-2);
				\end{tikzpicture}}}
		\]
		\[
		\vcenter{\hbox{\begin{tikzpicture}
				\matrix [matrix of math nodes] (m) {
					\trm{(N+aM)+b(N+aM)} &[30pt] \trm{N+aM}
					\\[20pt]\trm{(N+b(N+aM))+a(M+b(N+aM))} %&&[10pt] (a<b)
					\\[20pt]\trm{((N+bN)+a(N+bM))+a((M+bN)+a(M+bM))} & \trm{(N+bN)+a(M+bM)}
					\\ };
				\draw[rw] (m-1-1) --node[above]{$\idem$}    (m-1-2);
				\draw[rw] (m-1-1) --node[left] {$\plusL$} (m-2-1);
				\draw[rws,implied] (m-2-1) --node[left] {$\plusR$}   (m-3-1);
				\draw[rws,implied] (m-3-2) --node[right]{$\idem$}    (m-1-2);
				\draw[rws,implied] (m-3-1) --node[below]{$\cancelL,\cancelR$} (m-3-2);
				\end{tikzpicture}}}
		\]
	}
	
	%=====
	\itm\cancelL{\trm{(N +a M) +a P}\rw\trm{N +a P}}
	{\small
		\[
		\vcenter{\hbox{\begin{tikzpicture}
				\matrix [matrix of math nodes] (m) {
					\trm{(N+aM)+a(P+aQ)} &[15pt] \trm{N+a(P+aQ)}
					\\[20pt]\trm{(N+aM)+a Q} & \trm{N+aQ}
					\\ };
				\draw[rw] (m-1-1) --node[above]{$\cancelL$} (m-1-2);
				\draw[rw] (m-1-1) --node[left] {$\cancelR$} (m-2-1);
				\draw[rw,implied] (m-1-2) --node[right]{$\cancelR$} (m-2-2);
				\draw[rw,implied] (m-2-1) --node[below]{$\cancelL$} (m-2-2);
				\end{tikzpicture}}}
		\vcenter{\hbox{\begin{tikzpicture}
				\matrix [matrix of math nodes] (m) {
					\trm{C[(N+aM)+aP]} &[15pt] \trm{C[N+aM]}
					\\[20pt]\trm{C[N+aM]+aC[P]}
					\\[20pt]\trm{(C[N]+aC[M])+aC[P]} & \trm{C[N]+aC[M]}
					\\ };
				\draw[rw] (m-1-1) --node[above]{$\cancelL$} (m-1-2);
				\draw[rw] (m-1-1) --node[left] {$\plusX$}   (m-2-1);
				\draw[rw,implied] (m-2-1) --node[left] {$\plusX$}   (m-3-1);
				\draw[rw,implied] (m-1-2) --node[right]{$\plusX$}   (m-3-2);
				\draw[rw,implied] (m-3-1) --node[below]{$\cancelL$} (m-3-2);
				\end{tikzpicture}}}
		\]
		\[
		\vcenter{\hbox{\begin{tikzpicture}
				\matrix [matrix of math nodes] (m) {
					\trm{(N+bM)+b(P+aQ)} &[20pt]  \trm{N+b(P+aQ)}
					\\[20pt]\trm{((N+bM)+bP)+a((N+bM)+bQ)} & \trm{(N+bP)+a(N+bQ)}
					\\ };
				\draw[rw] (m-1-1) --node[above]{$\cancelL$} (m-1-2);
				\draw[rw] (m-1-1) --node[left] {$\plusR$}   (m-2-1);
				\draw[rw, implied] (m-1-2) --node[right]{$\plusR$} (m-2-2);
				\draw[rws,implied] (m-2-1) --node[below]{$\cancelL$} (m-2-2);
				\end{tikzpicture}}}
		%\qquad(a<b)
		\]
	}
	
	%=====
	\itm\plusX{\trm{C[N+aM]}\rw\trm{C[N]+a C[M]}}
	{\small
		\[
		\vcenter{\hbox{\begin{tikzpicture}
				\matrix [matrix of math nodes] (m) {
					\trm{C[(N+aM)+bP]}      &[20pt] \trm{C[N+aM]+bC[P]}
					\\[20pt]\trm{C[(N+bP)+a(M+bP)]} &       \trm{(C[N]+aC[M])+bC[P]}		 %&[10pt] (a<b)
					\\[20pt]\trm{C[N+bP]+aC[M+bP]}  &       \trm{(C[N]+bC[P])+a(C[M]+bC[P])}
					\\ };
				\draw[rw] (m-1-1) --node[above]{$\plusX$} (m-1-2);
				\draw[rw] (m-1-1) --node[left] {$\plusL$} (m-2-1);
				\draw[rw, implied] (m-1-2) --node[right]{$\plusX$} (m-2-2);
				\draw[rw, implied] (m-2-1) --node[left] {$\plusX$} (m-3-1);
				\draw[rw, implied] (m-2-2) --node[right]{$\plusL$} (m-3-2);
				\draw[rws,implied] (m-3-1) --node[below]{$\plusX$} (m-3-2);
				\end{tikzpicture}}}
		\]
		
		%=====
		\itm\plusL{\trm{(N+aM)+bP}\rw\trm{(N+bP)+a(M+bP)}}
		\[
		\vcenter{\hbox{\begin{tikzpicture}[x=340pt,y=40pt]
				\node[anchor=west] (a) at (0,2) {$\trm{(N+aM)+b(P+aQ)}$};
				\node[anchor=west] (b) at (0,1) {$\trm{((N+aM)+bP)+a((N+aM)+bQ)}$};
				\node[anchor=west] (c) at (0,0) {$\trm{((N+bP)+a(M+bP))+a((N+bQ)+a(M+bQ))}$};
				%
				\node[anchor=east] (d) at (1,2) {$\trm{(N+b(P+aQ))+a(M+b(P+aQ))}$};
				\node[anchor=east] (e) at (1,1) {$\trm{((N+bP)+a(N+bQ))+a((M+bP)+a(M+bQ))}$};
				\node[anchor=east] (f) at (1,0) {$\trm{(N+bP)+a(M+bQ)}$};
				%
				\draw[rw] (a) --node[above]{$\plusL$} (d);
				\draw[rw] 			($(a.south west)+( 43.5pt,0pt)$) --node[left] {$\plusR$} ($(b.north west)+( 43.5pt,0pt)$);
				\draw[rws,implied]	($(d.south east)+(-43.5pt,0pt)$) --node[right]{$\plusR$} ($(e.north east)+(-43.5pt,0pt)$);
				\draw[rws,implied]	($(b.south west)+( 43.5pt,0pt)$) --node[left] {$\plusL$} ($(c.north west)+( 43.5pt,0pt)$);
				\draw[rws,implied]	($(e.south east)+(-43.5pt,0pt)$) --node[right]{$\cancelL,\cancelR$} ($(f.north east)+(-43.5pt,0pt)$);
				\draw[rws,implied] (c) --node[below]{$\cancelL,\cancelR$} (f);
				\end{tikzpicture}}}
		%\vcenter{\hbox{\begin{tikzpicture}
		%	\matrix [matrix of math nodes] (m) {
		%	  		\trm{(N+aM)+b(P+aQ)}                      &[10pt] \trm{(N+b(P+aQ))+a(M+b(P+aQ))}
		%	\\[20pt]\trm{((N+aM)+bP)+a((N+aM)+bQ)}            &       \trm{((N+bP)+a(N+bQ))+a((M+bP)+a(M+bQ))} %&[10pt] (a<b)
		%	\\[20pt]\trm{((N+bP)+a(M+bP))+a((N+bQ)+a(M+bQ))}  &       \trm{(N+bP)+a(M+bQ)}
		%	\\ };
		%	\draw[rw] (m-1-1) --node[above]{$\plusL$} (m-1-2);
		%	\draw[rw] (m-1-1) --node[left] {$\plusR$} (m-2-1);
		%	\draw[rws,implied] (m-1-2) --node[right]{$\plusR$} (m-2-2);
		%	\draw[rws,implied] (m-2-1) --node[left] {$\plusL$} (m-3-1);
		%	\draw[rws,implied] (m-2-2) --node[right]{$\cancelL,\cancelR$} (m-3-2);
		%	\draw[rws,implied] (m-3-1) --node[below]{$\cancelL,\cancelR$} (m-3-2);
		%\end{tikzpicture}}}
		\]
		\[
		\vcenter{\hbox{\begin{tikzpicture}
				\matrix [matrix of math nodes] (m) {
					\trm{(N+bM)+c(P+aQ)}            &[20pt] \trm{(N+c(P+aQ))+b(M+c(P+aQ))}
					\\[20pt]                                &       \trm{((N+cP)+a(N+cQ))+b((M+cP)+a(M+cQ))} %&[10pt] (a<b<c)
					\\[20pt]\trm{((N+bM)+cP)+a((N+bM)+cQ)}  &       \trm{((N+cP)+b(M+cP))+a((N+cQ)+b(M+cQ))}
					\\ };
				\draw[rw] (m-1-1) --node[above]{$\plusL$} (m-1-2);
				\draw[rw] (m-1-1) --node[left] {$\plusR$} (m-3-1);
				\draw[rws,implied] (m-1-2) --node[right]{$\plusR$} (m-2-2);
				\draw[rws,implied] (m-2-2) --node[right]{$\plusL,\plusR,\cancelL,\cancelR$} (m-3-2);
				\draw[rws,implied] (m-3-1) --node[below]{$\plusL$} (m-3-2);
				\end{tikzpicture}}}
		\]
	}
	
	%=====
%	\SLV{The remaining cases are considered in the Appendix of \cite{arXiv}}{The remaining cases are considered in the proof in the Appendix (p.~\pageref{lemmaAppendix:confluence-perm}).}
%	\qedhere
	
	%\itm\plusFun{\trm{(N+aM)P}\rw\trm{(NP)+a(MP)}}
	%{\small
	%\[
	%\vcenter{\hbox{\begin{tikzpicture}
	%	\matrix [matrix of math nodes] (m) {
	%	  		\trm{(N+aM)(P+aQ)}                &[10pt] \trm{(N(P+aQ))+a(M(P+aQ))}
	%	\\[20pt]\trm{((N+aM)P)+a((N+aM)Q)}        &       \trm{((NP)+a(NQ))+a((MP)+a(MQ))}
	%	\\[20pt]\trm{((NP)+a(MP))+a((NQ)+a(MQ))}  &       \trm{(NP)+a(MQ)}
	%	\\ };
	%	\draw[rw] (m-1-1) --node[above]{$\plusFun$} (m-1-2);
	%	\draw[rw] (m-1-1) --node[left] {$\plusArg$} (m-2-1);
	%	\draw[rws,implied] (m-1-2) --node[right]{$\plusArg$} (m-2-2);
	%	\draw[rws,implied] (m-2-1) --node[left] {$\plusFun$} (m-3-1);
	%	\draw[rws,implied] (m-2-2) --node[right]{$\cancelL,\cancelR$} (m-3-2);
	%	\draw[rws,implied] (m-3-1) --node[below]{$\cancelL,\cancelR$} (m-3-2);
	%\end{tikzpicture}}}
	%\]
	%\[
	%\vcenter{\hbox{\begin{tikzpicture}
	%	\matrix [matrix of math nodes] (m) {
	%	  		\trm{(N+bM)(P+aQ)}         &[20pt] \trm{(N(P+aQ))+b(M(P+aQ))}
	%	\\[20pt]                           &       \trm{((NP)+a(NQ))+b((MP)+a(MQ))} %&[10pt] (a<b)
	%	\\[20pt]\trm{((N+bM)P)+a((N+bM)Q)} &       \trm{((NP)+b(MP))+a((NQ)+b(MQ))}
	%	\\ };
	%	\draw[rw] (m-1-1) --node[above]{$\plusFun$} (m-1-2);
	%	\draw[rw] (m-1-1) --node[left] {$\plusArg$} (m-3-1);
	%	\draw[rws,implied] (m-1-2) --node[right]{$\plusArg$} (m-2-2);
	%	\draw[rws,implied] (m-2-2) --node[right]{$\plusL,\plusR,\cancelL,\cancelR$} (m-3-2);
	%	\draw[rws,implied] (m-3-1) --node[below]{$\plusFun$} (m-3-2);
	%\end{tikzpicture}}}
	%\]
	%}
	%
	%%=====
	%\itm\plusBox{\trm{!b.(N +a M)} \rw \trm{(!b.N) +a (!b.M)}}
	%{\small
	%\[
	%\vcenter{\hbox{\begin{tikzpicture}
	%	\matrix [matrix of math nodes] (m) {
	%	  		\trm{!b.(N+aM)}    &[20pt] \trm{(!b.N)+a(!b.M)}
	%	\\[20pt]\trm{N+aM}
	%	\\ };
	%	\draw[rw] (m-1-1) --node[above]{$\plusBox$} (m-1-2);
	%	\draw[rw] (m-1-1) --node[left] {$\boxVoid$} (m-2-1);
	%	\draw[rws,implied] (m-1-2) --node[below right=-2pt] {$\boxVoid$} (m-2-1);
	%\end{tikzpicture}}}
	%\quad (a \neq b, \ b \notin\trm{N+a M})
	%\]
	%}
%\end{proof}

%\begin{proof}
%	We prove local confluence; by Newman's lemma and strong normalization of $\rw_\perm$ (\Cref{lemma:strong-normalization}), confluence follows. We consider each reduction rule against those lower down in Figure~\ref{fig:reduction rules}. For the symmetric rule pairs $\cancelL$/$\cancelR$, $\plusL$/$\plusR$, and $\plusFun/\plusArg$ we present only the first case. Unless otherwise specified, we let $a<b<c$.
%
%\newcommand\itm[2]{\medskip\noindent(#1)}% \qquad$#2$}
%
%	%=====
%	\itm\idem{\trm{N +a N}\rw N}
%	{\small
%		\[
%		\vc{\begin{tikzpicture}[x=20pt,y=1.5ex]
%			\node[anchor=base east] (a) at (0,0) {$\trm{(N+aN)+aM}$};
%			\node[anchor=base west] (b) at (1,0) {$\trm{N+aM}$};
%			\draw[rw] (0,1) --node[above]{$\idem$}    (1,1);
%			\draw[rw] (0,0) --node[below]{$\cancelL$} (1,0);
%			\end{tikzpicture}}
%		\]
%		\[
%		\vcenter{\hbox{\begin{tikzpicture}
%				\matrix [matrix of math nodes] (m) {
%					\trm{(N+aM)+a(N+aM)} &[20pt] \trm{N+aM}
%					\\[20pt]\trm{N+a(N+aM)}
%					\\ };
%				\draw[rw] (m-1-1) --node[above]{$\idem$}    (m-1-2);
%				\draw[rw] (m-1-1) --node[left] {$\cancelL$} (m-2-1);
%				\draw[rw,implied] (m-2-1) --node[below right=-2pt] {$\cancelR$} (m-1-2);
%				\end{tikzpicture}}}
%		\qquad
%		\vcenter{\hbox{\begin{tikzpicture}
%				\matrix [matrix of math nodes] (m) {
%					\trm{C[N+aN]}    &[20pt] C[N]
%					\\[20pt]\trm{C[N]+aC[N]}
%					\\ };
%				\draw[rw] (m-1-1) --node[above]{$\idem$}  (m-1-2);
%				\draw[rw] (m-1-1) --node[left] {$\plusX$} (m-2-1);
%				\draw[rw,implied] (m-2-1) --node[below right=-2pt] {$\idem$} (m-1-2);
%				\end{tikzpicture}}}
%		\]
%		\[
%		\vcenter{\hbox{\begin{tikzpicture}
%				\matrix [matrix of math nodes] (m) {
%					\trm{(N+aM)+b(N+aM)} &[30pt] \trm{N+aM}
%					\\[20pt]\trm{(N+b(N+aM))+a(M+b(N+aM))} %&&[10pt] (a<b)
%					\\[20pt]\trm{((N+bN)+a(N+bM))+a((M+bN)+a(M+bM))} & \trm{(N+bN)+a(M+bM)}
%					\\ };
%				\draw[rw] (m-1-1) --node[above]{$\idem$}    (m-1-2);
%				\draw[rw] (m-1-1) --node[left] {$\plusL$} (m-2-1);
%				\draw[rws,implied] (m-2-1) --node[left] {$\plusR$}   (m-3-1);
%				\draw[rws,implied] (m-3-2) --node[right]{$\idem$}    (m-1-2);
%				\draw[rws,implied] (m-3-1) --node[below]{$\cancelL,\cancelR$} (m-3-2);
%				\end{tikzpicture}}}
%		\]
%	}
%
%	%=====
%	\itm\cancelL{\trm{(N +a M) +a P}\rw\trm{N +a P}}
%	{\small
%		\[
%		\vcenter{\hbox{\begin{tikzpicture}
%				\matrix [matrix of math nodes] (m) {
%					\trm{(N+aM)+a(P+aQ)} &[15pt] \trm{N+a(P+aQ)}
%					\\[20pt]\trm{(N+aM)+a Q} & \trm{N+aQ}
%					\\ };
%				\draw[rw] (m-1-1) --node[above]{$\cancelL$} (m-1-2);
%				\draw[rw] (m-1-1) --node[left] {$\cancelR$} (m-2-1);
%				\draw[rw,implied] (m-1-2) --node[right]{$\cancelR$} (m-2-2);
%				\draw[rw,implied] (m-2-1) --node[below]{$\cancelL$} (m-2-2);
%				\end{tikzpicture}}}
%		\vcenter{\hbox{\begin{tikzpicture}
%				\matrix [matrix of math nodes] (m) {
%					\trm{C[(N+aM)+aP]} &[15pt] \trm{C[N+aM]}
%					\\[20pt]\trm{C[N+aM]+aC[P]}
%					\\[20pt]\trm{(C[N]+aC[M])+aC[P]} & \trm{C[N]+aC[M]}
%					\\ };
%				\draw[rw] (m-1-1) --node[above]{$\cancelL$} (m-1-2);
%				\draw[rw] (m-1-1) --node[left] {$\plusX$}   (m-2-1);
%				\draw[rw,implied] (m-2-1) --node[left] {$\plusX$}   (m-3-1);
%				\draw[rw,implied] (m-1-2) --node[right]{$\plusX$}   (m-3-2);
%				\draw[rw,implied] (m-3-1) --node[below]{$\cancelL$} (m-3-2);
%				\end{tikzpicture}}}
%		\]
%		\[
%		\vcenter{\hbox{\begin{tikzpicture}
%				\matrix [matrix of math nodes] (m) {
%					\trm{(N+bM)+b(P+aQ)} &[20pt]  \trm{N+b(P+aQ)}
%					\\[20pt]\trm{((N+bM)+bP)+a((N+bM)+bQ)} & \trm{(N+bP)+a(N+bQ)}
%					\\ };
%				\draw[rw] (m-1-1) --node[above]{$\cancelL$} (m-1-2);
%				\draw[rw] (m-1-1) --node[left] {$\plusR$}   (m-2-1);
%				\draw[rw, implied] (m-1-2) --node[right]{$\plusR$} (m-2-2);
%				\draw[rws,implied] (m-2-1) --node[below]{$\cancelL$} (m-2-2);
%				\end{tikzpicture}}}
%		%\qquad(a<b)
%		\]
%	}
%
%	%=====
%	\itm\plusX{\trm{C[N+aM]}\rw\trm{C[N]+a C[M]}}
%	{\small
%		\[
%		\vcenter{\hbox{\begin{tikzpicture}
%				\matrix [matrix of math nodes] (m) {
%					\trm{C[(N+aM)+bP]}      &[20pt] \trm{C[N+aM]+bC[P]}
%					\\[20pt]\trm{C[(N+bP)+a(M+bP)]} &       \trm{(C[N]+aC[M])+bC[P]}		 %&[10pt] (a<b)
%					\\[20pt]\trm{C[N+bP]+aC[M+bP]}  &       \trm{(C[N]+bC[P])+a(C[M]+bC[P])}
%					\\ };
%				\draw[rw] (m-1-1) --node[above]{$\plusX$} (m-1-2);
%				\draw[rw] (m-1-1) --node[left] {$\plusL$} (m-2-1);
%				\draw[rw, implied] (m-1-2) --node[right]{$\plusX$} (m-2-2);
%				\draw[rw, implied] (m-2-1) --node[left] {$\plusX$} (m-3-1);
%				\draw[rw, implied] (m-2-2) --node[right]{$\plusL$} (m-3-2);
%				\draw[rws,implied] (m-3-1) --node[below]{$\plusX$} (m-3-2);
%				\end{tikzpicture}}}
%		\]
%
%		%=====
%		\itm\plusL{\trm{(N+aM)+bP}\rw\trm{(N+bP)+a(M+bP)}}
%		\[
%		\vcenter{\hbox{\begin{tikzpicture}[x=340pt,y=40pt]
%				\node[anchor=west] (a) at (0,2) {$\trm{(N+aM)+b(P+aQ)}$};
%				\node[anchor=west] (b) at (0,1) {$\trm{((N+aM)+bP)+a((N+aM)+bQ)}$};
%				\node[anchor=west] (c) at (0,0) {$\trm{((N+bP)+a(M+bP))+a((N+bQ)+a(M+bQ))}$};
%				%
%				\node[anchor=east] (d) at (1,2) {$\trm{(N+b(P+aQ))+a(M+b(P+aQ))}$};
%				\node[anchor=east] (e) at (1,1) {$\trm{((N+bP)+a(N+bQ))+a((M+bP)+a(M+bQ))}$};
%				\node[anchor=east] (f) at (1,0) {$\trm{(N+bP)+a(M+bQ)}$};
%				%
%				\draw[rw] (a) --node[above]{$\plusL$} (d);
%				\draw[rw] 			($(a.south west)+( 43.5pt,0pt)$) --node[left] {$\plusR$} ($(b.north west)+( 43.5pt,0pt)$);
%				\draw[rws,implied]	($(d.south east)+(-43.5pt,0pt)$) --node[right]{$\plusR$} ($(e.north east)+(-43.5pt,0pt)$);
%				\draw[rws,implied]	($(b.south west)+( 43.5pt,0pt)$) --node[left] {$\plusL$} ($(c.north west)+( 43.5pt,0pt)$);
%				\draw[rws,implied]	($(e.south east)+(-43.5pt,0pt)$) --node[right]{$\cancelL,\cancelR$} ($(f.north east)+(-43.5pt,0pt)$);
%				\draw[rws,implied] (c) --node[below]{$\cancelL,\cancelR$} (f);
%				\end{tikzpicture}}}
%		%\vcenter{\hbox{\begin{tikzpicture}
%		%	\matrix [matrix of math nodes] (m) {
%		%	  		\trm{(N+aM)+b(P+aQ)}                      &[10pt] \trm{(N+b(P+aQ))+a(M+b(P+aQ))}
%		%	\\[20pt]\trm{((N+aM)+bP)+a((N+aM)+bQ)}            &       \trm{((N+bP)+a(N+bQ))+a((M+bP)+a(M+bQ))} %&[10pt] (a<b)
%		%	\\[20pt]\trm{((N+bP)+a(M+bP))+a((N+bQ)+a(M+bQ))}  &       \trm{(N+bP)+a(M+bQ)}
%		%	\\ };
%		%	\draw[rw] (m-1-1) --node[above]{$\plusL$} (m-1-2);
%		%	\draw[rw] (m-1-1) --node[left] {$\plusR$} (m-2-1);
%		%	\draw[rws,implied] (m-1-2) --node[right]{$\plusR$} (m-2-2);
%		%	\draw[rws,implied] (m-2-1) --node[left] {$\plusL$} (m-3-1);
%		%	\draw[rws,implied] (m-2-2) --node[right]{$\cancelL,\cancelR$} (m-3-2);
%		%	\draw[rws,implied] (m-3-1) --node[below]{$\cancelL,\cancelR$} (m-3-2);
%		%\end{tikzpicture}}}
%		\]
%		\[
%		\vcenter{\hbox{\begin{tikzpicture}
%				\matrix [matrix of math nodes] (m) {
%					\trm{(N+bM)+c(P+aQ)}            &[20pt] \trm{(N+c(P+aQ))+b(M+c(P+aQ))}
%					\\[20pt]                                &       \trm{((N+cP)+a(N+cQ))+b((M+cP)+a(M+cQ))} %&[10pt] (a<b<c)
%					\\[20pt]\trm{((N+bM)+cP)+a((N+bM)+cQ)}  &       \trm{((N+cP)+b(M+cP))+a((N+cQ)+b(M+cQ))}
%					\\ };
%				\draw[rw] (m-1-1) --node[above]{$\plusL$} (m-1-2);
%				\draw[rw] (m-1-1) --node[left] {$\plusR$} (m-3-1);
%				\draw[rws,implied] (m-1-2) --node[right]{$\plusR$} (m-2-2);
%				\draw[rws,implied] (m-2-2) --node[right]{$\plusL,\plusR,\cancelL,\cancelR$} (m-3-2);
%				\draw[rws,implied] (m-3-1) --node[below]{$\plusL$} (m-3-2);
%				\end{tikzpicture}}}
%		\]
%	}
%
%	%=====

%	We consider only the cases omitted in the proof on p.~\pageref{lem:confluence-perm}.

	\itm\plusFun{\trm{(N+aM)P}\rw\trm{(NP)+a(MP)}}
	{\small
		\[
		\vcenter{\hbox{\begin{tikzpicture}
				\matrix [matrix of math nodes] (m) {
					\trm{(N+aM)(P+aQ)}                &[10pt] \trm{(N(P+aQ))+a(M(P+aQ))}
					\\[20pt]\trm{((N+aM)P)+a((N+aM)Q)}        &       \trm{((NP)+a(NQ))+a((MP)+a(MQ))}
					\\[20pt]\trm{((NP)+a(MP))+a((NQ)+a(MQ))}  &       \trm{(NP)+a(MQ)}
					\\ };
				\draw[rw] (m-1-1) --node[above]{$\plusFun$} (m-1-2);
				\draw[rw] (m-1-1) --node[left] {$\plusArg$} (m-2-1);
				\draw[rws,implied] (m-1-2) --node[right]{$\plusArg$} (m-2-2);
				\draw[rws,implied] (m-2-1) --node[left] {$\plusFun$} (m-3-1);
				\draw[rws,implied] (m-2-2) --node[right]{$\cancelL,\cancelR$} (m-3-2);
				\draw[rws,implied] (m-3-1) --node[below]{$\cancelL,\cancelR$} (m-3-2);
				\end{tikzpicture}}}
		\]
		\[
		\vcenter{\hbox{\begin{tikzpicture}
				\matrix [matrix of math nodes] (m) {
					\trm{(N+bM)(P+aQ)}         &[20pt] \trm{(N(P+aQ))+b(M(P+aQ))}
					\\[20pt]                           &       \trm{((NP)+a(NQ))+b((MP)+a(MQ))} %&[10pt] (a<b)
					\\[20pt]\trm{((N+bM)P)+a((N+bM)Q)} &       \trm{((NP)+b(MP))+a((NQ)+b(MQ))}
					\\ };
				\draw[rw] (m-1-1) --node[above]{$\plusFun$} (m-1-2);
				\draw[rw] (m-1-1) --node[left] {$\plusArg$} (m-3-1);
				\draw[rws,implied] (m-1-2) --node[right]{$\plusArg$} (m-2-2);
				\draw[rws,implied] (m-2-2) --node[right]{$\plusL,\plusR,\cancelL,\cancelR$} (m-3-2);
				\draw[rws,implied] (m-3-1) --node[below]{$\plusFun$} (m-3-2);
				\end{tikzpicture}}}
		\]
	}

	%=====
	\itm\plusBox{\trm{!b.(N +a M)} \rw \trm{(!b.N) +a (!b.M)}}
	{\small
		\[
		\vcenter{\hbox{\begin{tikzpicture}
				\matrix [matrix of math nodes] (m) {
					\trm{!b.(N+aM)}    &[20pt] \trm{(!b.N)+a(!b.M)}
					\\[20pt]\trm{N+aM}
					\\ };
				\draw[rw] (m-1-1) --node[above]{$\plusBox$} (m-1-2);
				\draw[rw] (m-1-1) --node[left] {$\boxVoid$} (m-2-1);
				\draw[rws,implied] (m-1-2) --node[below right=-2pt] {$\boxVoid$} (m-2-1);
				\end{tikzpicture}}}
		\quad (a \neq b, \ b \notin\trm{N+a M})
		\]
	}
\end{proof}

%\setcounter{lemmaAppendix}{\value{lemma:application-parallel-beta}}
%\begin{lemmaAppendix}
%	\label{lemmaAppendix:application-parallel-beta}
%	If $\trm{M} \rwp_\beta \trm{M'}$ and $\trm{N} \rwp_\beta \trm{N'}$,
%%	\marginpar{\footnotesize Proof in the Appendix}
%	then $\trm{MN} \rwp_\beta \trm{M'N'}$.
%	If moreover $M = \lambda x.R$ and $M'^ = \lambda x R'$ with $R \rwp_\beta R'$, then  $\trm{MN} \rwp_\beta \trm{R'[N'/x]}$.
%\end{lemmaAppendix}
%
%\begin{proof}
%	Since $\trm{M} \rwp_\beta \trm{Mo} = M'$ and $\trm{N} \rwp_\beta \trm{No} = N'$ for some labelings $\trm{M*}$ and $\trm{N*}$ of $M$ and $N$ respectively, let $\trm{(MN)*}$ be the labeling of $\trm{MN}$ obtained by putting together the labelings $\trm{M*}$ and $\trm{N*}$.
%	Clearly, $\trm{MN} \rwp_\beta \trm{<(MN)*>} = \trm{MoNo} = M'N'$.
%
%	Concerning the part of the statement after ``moreover'', since $\trm{R} \rwp_\beta \trm{<R*>} = R'$ and $\trm{N} \rwp_\beta \trm{No} = N'$ for some labelings $\trm{R*}$ and $\trm{N*}$ of $R$ and $N$ respectively, let $\trm{(MN)*}$ be the labeling of $\trm{MN}$ obtained by putting together the labelings $\trm{R*}$ and $\trm{N*}$ plus the labeled $\beta$-redex $\trm{(\x.R)*N}$.
%	Clearly, $\trm{MN} \rwp_\beta \trm{<(MN)*>} = \trm{<R*>[No/x]} = \trm{R'[N'/x]}$.
%\end{proof}

\setcounter{lemmaAppendix}{\value{lemma:cloredbox}}
\begin{lemmaAppendix}\label{lemmaAppendix:cloredbox}
	The following rule is sound:
	$$
	\infer{\trm{(!a.M)}L_1\ldots L_m\in\mathit{SN}^\theta}{\trm{M}L_1\ldots L_m\in\mathit{SN}^{a\cdot\theta} & \forall i.a\not\in L_i}
	$$
\end{lemmaAppendix}
\begin{proof}
	The proof is structurally very similar to the one of
	Lemma~\ref{lemma:cloredsum}. The lexicographic order
	is however the following one:
	$$
	(m,\sum_{i=1}^m\mathit{sn}^{a\cdot\theta}(L_i)+\mathit{sn}^{a\cdot\theta}(M),|M|)
	$$
	There is one case in which we need to use
	Lemma \ref{lemma:cloredsum}, namely the one in which the
	considered reduction rule is the following:
	$$
	\trm{!b.(P +a Q)}\rw_\perm\trm{(!b.P) +a (!b.Q)}.
	$$
	Since $M=\trm{P +a Q}$ and $ML_1\ldots L_m$ by hypothesis,
	we conclusde that $PL_1\ldots L_m$ and $QL_1\ldots L_m$
	are both strongly normalizing. By induction hypothesis,
	since $\mathit{sn}^{a\cdot\theta}(P),\mathit{sn}^{a\cdot\theta}(Q)\leq\mathit{sn}^{a\cdot\theta}(M)$,
	it holds that $\trm{(!b.P)}L_1\ldots L_m$ and $\trm{(!b.Q)}L_1\ldots L_m$
	are both strongly normalizing themselves. Lemma~\ref{lemma:cloredsum},
	yields the thesis.
\end{proof}

%\begin{lem}\label{lemma:cloredheadvar}
%	The following rule is sound
%	$$
%	\infer{xL_1\ldots L_m\in\mathit{SN}^\theta}{L_1\in\mathit{SN}^\theta &\cdots & L_m\in\mathit{SN}^\theta}
%	$$
%\end{lem}
%
%\begin{proof}
%	Trivial, since the term $xL_1\ldots L_m$ cannot create new redexes.
%\end{proof}

\setcounter{lemmaAppendix}{\value{lemma:cloredbeta}}
\begin{lemmaAppendix}\label{lemmaAppendix:cloredbeta}
	The following rule is sound
	$$
	\infer{\trm{(\x.M)}L_0\ldots L_m\in\mathit{SN}^\theta}{\trm{M[L_0/x]}L_1\ldots L_m\in\mathit{SN}^\theta & L_0\in\mathit{SN}^\theta}
	$$
\end{lemmaAppendix}
\begin{proof}
	Again, the proof is structurally very similar to the one
	of Lemma~\ref{lemma:cloredsum}. The underlying order,
	needs to be slightly adapted, and is the lexicographic
	order on
	\begin{equation}\label{equ:redordbet}
	(\mathit{sn}(\trm{M[L_0/x]}L_1\ldots L_m)+\mathit{sn}(L_0),|M|)
	\end{equation}
	As usual, we proceed by showing that all terms to which
	$\trm{(\x.M)}L_0\ldots L_m$ reduces are strongly normalizing:
	\begin{itemize}
		\item
		If reduction happens in $L_0$, then we can mimick
		the same reduction in by zero or more reduction
		steps in $\trm{M[L_0/x]}L_1\ldots L_m$, and conclude
		by induction hypothesis, because the first component
		of (\ref{equ:redordbet}) strictly decreases.
		\item
		If reduction happens in $M$ or in $L_1,\ldots,L_m$,
		then we can mimick the same reduction in one or more
		reduction in $\trm{M[L_0/x]}L_1\ldots L_m$, and conclude
		by induction hypothesis since, again, the first component
		of (\ref{equ:redordbet}) strictly decreases.
		\item
		If the reduction step we perform reduces
		$\trm{(\x.M)}L_0$, then the thesis follows from the
		hypothesis about $\trm{M[L_0/x]}L_1\ldots L_m$.
		\item
		If $M$ is in the form $\trm{P +a Q}$ and
		the reduction step we perform reduces
		$\trm{(\x.M)}L_0\ldots L_m$
		to $\trm{((\x.P)+a(\x.Q))}L_0\ldots L_m$,
		we proceed by observing that
		$\trm{(\x.P)}L_0\ldots L_m$
		and $\trm{(\x.Q)}L_0\ldots L_m$ are
		both in $\mathit{SN}^\theta$ and we
		can apply the induction hypothesis to them,
		because the first component of (\ref{equ:redordbet})
		stays the same, but the second one strictly decreases.
		We then obtain that
		$\trm{P[x/L_0])}L_1\ldots L_m$
		and $\trm{Q[x/L_0]}L_1\ldots L_m$ are both
		in $\mathit{SN}^\theta$, and from Lemma~\ref{lemma:cloredsum}
		we get the thesis.
		\item
		If $M$ is in the form $\trm{!a.P}$ and
		the reduction step we perform reduces
		$\trm{(\x.M)}L_0\ldots L_m$ to
		$\trm{!a.(\x.P)}L_0\ldots L_m$. We can
		first of all that we can assume that
		$a\not\in L_i$ for every $0\leq i\leq m$.
		We proceed by observing that
		$\trm{(\x.P)}L_0\ldots L_m$ is in $\mathit{SN}^\theta$
		and we can apply the \ih\ to it, because
		the first component of (\ref{equ:redordbet})
		stays the same, but the second one strictly decreases.
		We then obtain that
		$\trm{P[x/L_0]}L_1\ldots L_m$
		is in $\mathit{SN}^\theta$, and from Lemma~\ref{lemma:cloredbox}
		we get the thesis.
		\qedhere
	\end{itemize}
\end{proof}
